\HeaderA{BrainSlice-class}{Class "BrainSlice" for two-dimensional image data}{BrainSlice.Rdash.class}
\aliasA{indexToVoxel,BrainSlice,index-method}{BrainSlice-class}{indexToVoxel,BrainSlice,index.Rdash.method}
\aliasA{numdim,BrainSlice-method}{BrainSlice-class}{numdim,BrainSlice.Rdash.method}
\aliasA{voxelToIndex,BrainSlice,index,index-method}{BrainSlice-class}{voxelToIndex,BrainSlice,index,index.Rdash.method}
\keyword{classes}{BrainSlice-class}
\begin{Description}\relax
This class is used to store two-dimensional image data
representing a single cut or slice through a brain volume
\end{Description}
\begin{Section}{Objects from the Class}
Objects can be created by calls of the form \code{ BrainSlice(data, space)}
\end{Section}
\begin{Section}{Slots}
\describe{
\item[\code{.Data}:] Object of class \code{"array"} inherited data slot
from "array" class. In this class, however, array data is always
two-dimensional 

\item[\code{space}:] Object of class \code{"BrainSpace"} inherited
slot from virtual class "BrainData" 
}
\end{Section}
\begin{Section}{Extends}
Class \code{"\LinkA{array}{array.Rdash.class}"}, from data part.
Class \code{"\LinkA{BrainData}{BrainData.Rdash.class}"}, directly.
Class \code{"\LinkA{structure}{structure.Rdash.class}"}, by class "array", distance 2.
Class \code{"\LinkA{matrix}{matrix.Rdash.class}"}, by class "array", distance 2, with explicit test and coerce.
Class \code{"\LinkA{vector}{vector.Rdash.class}"}, by class "array", distance 3, with explicit coerce.
Class \code{"\LinkA{vector}{vector.Rdash.class}"}, by class "array", distance 4, with explicit test and coerce.
Class \code{"\LinkA{vector}{vector.Rdash.class}"}, by class "array", distance 5, with explicit test and coerce.
\end{Section}
\begin{Section}{Methods}
\describe{
\item[indexToVoxel] \code{signature(x = "BrainSlice", idx =
        "index")}: -- maps from a one-dimensional index to a
two-dimensional pixel/voxel coordinate 

\item[voxelToIndex] \code{signature(x = "BrainSlice", i = "index", j
        = "index")}: -- convert from a two-dimensional coordinate to
one-dimensional index 
}
\end{Section}
\begin{Author}\relax
Bradley Buchsbaum \email{bbuchsbaum@berkeley.edu}
\end{Author}
\begin{Examples}
\begin{ExampleCode}

#space <- BrainSpace(c(50,50))
#slice <- BrainSlice(rnorm(50*50), space)

\end{ExampleCode}
\end{Examples}


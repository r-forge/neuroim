\HeaderA{BrainData-class}{Virtual Class "BrainData"}{BrainData.Rdash.class}
\aliasA{dim,BrainData-method}{BrainData-class}{dim,BrainData.Rdash.method}
\aliasA{numdim,BrainData-method}{BrainData-class}{numdim,BrainData.Rdash.method}
\aliasA{space,BrainData-method}{BrainData-class}{space,BrainData.Rdash.method}
\aliasA{spacing,BrainData-method}{BrainData-class}{spacing,BrainData.Rdash.method}
\keyword{classes}{BrainData-class}
\begin{Description}\relax
The \code{BrainData} class is a class contained by all actual
classes in the \pkg{Neuroimage} package.  It is a \dQuote{virtual} class.
\end{Description}
\begin{Section}{Objects from the Class}
A virtual Class: No objects may be created from it.
\end{Section}
\begin{Section}{Slots}
Common to \emph{all} BrainData objects in the package:
\describe{

\item[\code{space}:] Object of class \code{"BrainSpace"} -- a
description of the data "space", e.g. voxel resolution and iamge
dimensions.

}
\end{Section}
\begin{Section}{Methods}
\describe{
\item[dim] \code{signature(x = "BrainData")}: extract data dimensions
\code{\LinkA{dim}{dim}}.
\item[space] \code{signature(x = "BrainData")}: extract
\dQuote{space} slot.
\item[numdim] \code{signature(x = "BrainData")}: return number of
dimensions in space 
\item[spacing] \code{signature(x = "BrainData")}: returns cell
spacing for image data  
}
\end{Section}
\begin{Note}\relax
No notes yet
\end{Note}
\begin{Author}\relax
Bradley Buchsbaum \email{bbuchsbaum@berkeley.edu}
\end{Author}
\begin{Examples}
\begin{ExampleCode}

slotNames("BrainData")

cl <- getClass("BrainData")
names(cl@subclasses) 

showClass("BrainData") # output with slots and all subclasses

\end{ExampleCode}
\end{Examples}


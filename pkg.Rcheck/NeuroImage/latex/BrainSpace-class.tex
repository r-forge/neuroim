\HeaderA{BrainSpace-class}{Class "BrainSpace"}{BrainSpace.Rdash.class}
\aliasA{bounds,BrainSpace-method}{BrainSpace-class}{bounds,BrainSpace.Rdash.method}
\aliasA{dim,BrainSpace-method}{BrainSpace-class}{dim,BrainSpace.Rdash.method}
\aliasA{invTrans,BrainSpace-method}{BrainSpace-class}{invTrans,BrainSpace.Rdash.method}
\aliasA{numdim,BrainSpace-method}{BrainSpace-class}{numdim,BrainSpace.Rdash.method}
\aliasA{orientation,BrainSpace-method}{BrainSpace-class}{orientation,BrainSpace.Rdash.method}
\aliasA{origin,BrainSpace-method}{BrainSpace-class}{origin,BrainSpace.Rdash.method}
\aliasA{reptime,BrainSpace-method}{BrainSpace-class}{reptime,BrainSpace.Rdash.method}
\aliasA{show,BrainSpace-method}{BrainSpace-class}{show,BrainSpace.Rdash.method}
\aliasA{spacing,BrainSpace-method}{BrainSpace-class}{spacing,BrainSpace.Rdash.method}
\aliasA{trans,BrainSpace-method}{BrainSpace-class}{trans,BrainSpace.Rdash.method}
\aliasA{trans<-,BrainSpace,matrix-method}{BrainSpace-class}{trans<.Rdash.,BrainSpace,matrix.Rdash.method}
\keyword{classes}{BrainSpace-class}
\begin{Description}\relax
BrainSpace represents the topology of a dataset
\end{Description}
\begin{Section}{Objects from the Class}
Objects can be created by calls of the form \code{BrainSpace(Dim,
  dimension, spacing, origin) }.
\end{Section}
\begin{Section}{Slots}
\describe{
\item[\code{Dim}:] Object of class \code{"integer"} --
an integer array specifiying the spatial dimensions 

\item[\code{origin}:] Object of class \code{"numeric"} --
an array representing the absolute location of the origin
(coordinate at voxel [0, 0, 0]) 

\item[\code{spacing}:] Object of class \code{"numeric"} --
an array representing the grid spacing for each dimension       

\item[\code{orientation}:] Object of class \code{"character"} --
character array describing axes orientation, e.g., "L" for
"left-right", "P" for "posterior-anterior", "I" for
"inferior-superior". This slot is currently not used and may
contain invalid information.

\item[\code{trans}:] Object of class \code{"matrix"} --
This is a transformation matrix used for computing the mapping
from index coordinates to real world oordiantes. Should be a 4X4
matrix for volumetrix data.

\item[\code{invTrans}:] Object of class \code{"matrix"} --
This is the inverse transformation (real world to index) and is
computed internally and therefore should not be changed externally 


\item[\code{reptime}:] Object of class \code{"numeric"} --
The delay between volume acquisitions in seconds 

}
\end{Section}
\begin{Section}{Methods}
\describe{
\item[bounds] \code{signature(x = "BrainSpace")}: -- returns the real
world bounds as a matrix of (min, max) pairs 

\item[invTrans] \code{signature(x = "BrainSpace")}: -- returns the
dimension of the data axes for the space 

\item[numdim] \code{signature(x = "BrainSpace")}: -- returns the
number of dimensions in space 

\item[orientation] \code{signature(x = "BrainSpace")}: -- returns
the (currently unsupported) orientation code  

\item[origin] \code{signature(x = "BrainSpace")}: -- returns the
coordinate origin 

\item[spacing] \code{signature(x = "BrainSpace")}: -- returns the voxel
spacing 

\item[trans\textless{}-] \code{signature(x = "BrainSpace", value = "matrix")}:
-- returns the index-to-word transformation matrix 

\item[trans] \code{signature(x = "BrainSpace")}: -- returns the
inverse transformation matrix 

\item[reptime] \code{signature(x = "BrainSpace")}: -- returns the
repetition time 

\item[show] \code{signature(x = "BrainSpace")}: -- print information
about BrainSpace object 

}
\end{Section}
\begin{Author}\relax
Bradley Buchsbaum \email{bbuchsbaum@berkeley.edu}
\end{Author}
\begin{References}\relax
no references
\end{References}
\begin{Examples}
\begin{ExampleCode}

space <- BrainSpace(Dim=c(64,64,18))
show(space)
trans(space)
origin(space)

\end{ExampleCode}
\end{Examples}


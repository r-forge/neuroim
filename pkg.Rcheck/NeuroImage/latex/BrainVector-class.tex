\HeaderA{BrainVector-class}{Class "BrainVector"}{BrainVector.Rdash.class}
\keyword{classes}{BrainVector-class}
\begin{Description}\relax
a class for storing an array or sequence of image volumes,
such as fMRI data
\end{Description}
\begin{Section}{Objects from the Class}
Objects can be created by calls of the form \code{new("BrainVector", ...)}.
\end{Section}
\begin{Section}{Slots}
\describe{
\item[\code{.Data}:] Object of class \code{"array"} : the stored
image data 
\item[\code{space}:] Object of class \code{"BrainSpace"} : the
geometry of the image volumes, which must be identical for images in the sequence 
}
\end{Section}
\begin{Section}{Extends}
Class \code{"\LinkA{BrainVolume}{BrainVolume.Rdash.class}"}, directly.
Class \code{"\LinkA{BrainSlice}{BrainSlice.Rdash.class}"}, by class "BrainVolume", distance 2.
Class \code{"\LinkA{array}{array.Rdash.class}"}, by class "BrainVolume", distance 3.
Class \code{"\LinkA{BrainData}{BrainData.Rdash.class}"}, by class "BrainVolume", distance 3.
Class \code{"\LinkA{structure}{structure.Rdash.class}"}, by class "BrainVolume", distance 4.
Class \code{"\LinkA{matrix}{matrix.Rdash.class}"}, by class "BrainVolume", distance 4, with explicit test and coerce.
Class \code{"\LinkA{vector}{vector.Rdash.class}"}, by class "BrainVolume", distance 5, with explicit coerce.
Class \code{"\LinkA{vector}{vector.Rdash.class}"}, by class "BrainVolume", distance 6, with explicit test and coerce.
Class \code{"\LinkA{vector}{vector.Rdash.class}"}, by class "BrainVolume", distance 7, with explicit test and coerce.
\end{Section}
\begin{Section}{Methods}
No methods defined with class "BrainVector" in the signature.
\end{Section}


\HeaderA{orientation}{function returns the orientation of a brain image}{orientation}
\keyword{manip}{orientation}
\begin{Description}\relax
generic function that returns a character string coding the
orientation of a brain image
\end{Description}
\begin{Usage}
\begin{verbatim}
orientation(x)
\end{verbatim}
\end{Usage}
\begin{Arguments}
\begin{ldescription}
\item[\code{x}] an object deriving from class "BrainSpace" or "BrainData" 
\end{ldescription}
\end{Arguments}
\begin{Details}\relax
The orientation string encodes the anatomical direction of each of the
(spatial) image axes.  Thus, a standard "neurologically" oriented
image volume has the orientatrion string "LPI", whcih stands for
"Left, Posterior, Inferior".  Currently this function is used only for
information.
\end{Details}
\begin{Value}
character string representing image orientation
\end{Value}


\HeaderA{BrainFile-class}{"BrainFile" virtual class}{BrainFile.Rdash.class}
\aliasA{fileName,BrainFile-method}{BrainFile-class}{fileName,BrainFile.Rdash.method}
\aliasA{fileType,BrainFile-method}{BrainFile-class}{fileType,BrainFile.Rdash.method}
\aliasA{openFor,BrainFile-method}{BrainFile-class}{openFor,BrainFile.Rdash.method}
\aliasA{path,BrainFile-method}{BrainFile-class}{path,BrainFile.Rdash.method}
\aliasA{show,BrainFile-method}{BrainFile-class}{show,BrainFile.Rdash.method}
\keyword{classes}{BrainFile-class}
\begin{Description}\relax
A class that encapsulates function associated with brain image input/output
\end{Description}
\begin{Section}{Objects from the Class}
A virtual Class: No objects may be created from it.
\end{Section}
\begin{Section}{Slots}
\describe{
\item[\code{path}:] Object of class \code{"character"} : the path
to the image file   
\item[\code{open}:] Object of class \code{"character"} : whether the
file to be opened for read "r" or write  "w" 

}
\end{Section}
\begin{Section}{Methods}
\describe{  
\item[openFor] \code{signature(x = "BrainFile")}: querues whether
file is open for read or write 
\item[path] \code{signature(x = "BrainFile")}: the full path of the
file  
\item[show] \code{signature(x = "BrainFile")}: print a text
representation of the \code{BrainFile} object 
\item[fileName] \code{signature(x = "BrainFile")}: return the file
name of the current connection 
\item[fileType] \code{signature(x = "BrainFile")}: return the file
type of the backing file (e.g. NIFTI) 

}
\end{Section}


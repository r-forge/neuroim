\HeaderA{BrainList}{creates a new "BrainList" object}{BrainList}
\keyword{manip}{BrainList}
\begin{Description}\relax
function to create a new object of class "BrainList"
\end{Description}
\begin{Usage}
\begin{verbatim}
BrainList(volList, labels = NULL, files = NULL)
\end{verbatim}
\end{Usage}
\begin{Arguments}
\begin{ldescription}
\item[\code{volList}] the list of \code{BrainData} objects 
\item[\code{labels}] the names of the objects, optional 
\item[\code{files}] the file names of the images, optional 
\end{ldescription}
\end{Arguments}
\begin{Details}\relax
This class is used for storing related collections of brain images. It
is simply a class extending \code{list} and \code{BrainData}.
For instance, one may represent a set of MRIs across a group of
subjects, each having the same geometry, as a "BrainList". Unlike,
\code{BrainVector} instances, which are stored as arrays and
suitable for representing multi-dimensional images such as FMRI
time-series, \code{BrainList} objects are better suited for loose,
possibly unordered collections of related data sets. Thus, "BrainList"
objects are closely related to "bucket" data sets in AFNI.
\end{Details}
\begin{Value}
an object of class "BrainList"
\end{Value}


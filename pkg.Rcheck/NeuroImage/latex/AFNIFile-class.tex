\HeaderA{AFNIFile-class}{Class "AFNIFile"}{AFNIFile.Rdash.class}
\keyword{classes}{AFNIFile-class}
\begin{Description}\relax
a class that handles input/output of NIFTI formatted image
data
\end{Description}
\begin{Section}{Objects from the Class}
Objects can be created by calls of the form \code{new("AFNIFile", ...)}.
\end{Section}
\begin{Section}{Slots}
\describe{
\item[\code{path}:] Object of class \code{"character"} the file path 
\item[\code{open}:] Object of class \code{"character"} whether to
open file for reading "r" or writing "w"  
\item[\code{fileType}:] Object of class \code{"character"} : the
specific file type fopr this format 
}
\end{Section}
\begin{Section}{Extends}
Class \code{"\LinkA{BrainFile}{BrainFile.Rdash.class}"}, directly.
\end{Section}
\begin{Section}{Methods}
No methods defined with class "AFNIFile" in the signature.
\end{Section}


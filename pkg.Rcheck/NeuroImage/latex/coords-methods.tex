\HeaderA{coords-methods}{Methods for Function coords}{coords.Rdash.methods}
\aliasA{coords}{coords-methods}{coords}
\aliasA{coords,BrainRegion3D-method}{coords-methods}{coords,BrainRegion3D.Rdash.method}
\aliasA{coords,IndexLookupVolume-method}{coords-methods}{coords,IndexLookupVolume.Rdash.method}
\aliasA{coords,SparseBrainVector-method}{coords-methods}{coords,SparseBrainVector.Rdash.method}
\aliasA{coords,TiledBrainVector-method}{coords-methods}{coords,TiledBrainVector.Rdash.method}
\keyword{methods}{coords-methods}
\begin{Description}\relax
extracts the coordinates of a a coordnate-based image data structure as a matrix
\end{Description}
\begin{Section}{Methods}
\describe{

\item[x = "BrainRegion3D"] returns the coordinates comprising the region of interest 

\item[x = "IndexLookupVolume"] returns the coordinate corresponding to the indices in the volume 

\item[x = "SparseBrainVector"] returns the coordinates corresponding to the indices in the volume 

\item[x = "TiledBrainVector"] not currently implemented 
}
\end{Section}


\HeaderA{BrainVolume-class}{Class "BrainVolume"}{BrainVolume.Rdash.class}
\aliasA{writeVolume,BrainVolume,character-method}{BrainVolume-class}{writeVolume,BrainVolume,character.Rdash.method}
\keyword{classes}{BrainVolume-class}
\begin{Description}\relax
a class for storing 3D volumetric image data
\end{Description}
\begin{Section}{Objects from the Class}
Objects can be created by calls of the form \code{new("BrainVolume", ...)}
\end{Section}
\begin{Section}{Slots}
\describe{
\item[\code{.Data}:] Object of class \code{"array"} : the image data
array 
\item[\code{space}:] Object of class \code{"BrainSpace"} : the image
geometry 
}
\end{Section}
\begin{Section}{Extends}
Class \code{"\LinkA{BrainSlice}{BrainSlice.Rdash.class}"}, directly.
Class \code{"\LinkA{array}{array.Rdash.class}"}, by class "BrainSlice", distance 2.
Class \code{"\LinkA{BrainData}{BrainData.Rdash.class}"}, by class "BrainSlice", distance 2.
Class \code{"\LinkA{structure}{structure.Rdash.class}"}, by class "BrainSlice", distance 3.
Class \code{"\LinkA{matrix}{matrix.Rdash.class}"}, by class "BrainSlice", distance 3, with explicit test and coerce.
Class \code{"\LinkA{vector}{vector.Rdash.class}"}, by class "BrainSlice", distance 4, with explicit coerce.
Class \code{"\LinkA{vector}{vector.Rdash.class}"}, by class "BrainSlice", distance 5, with explicit test and coerce.
Class \code{"\LinkA{vector}{vector.Rdash.class}"}, by class "BrainSlice", distance 6, with explicit test and coerce.
\end{Section}
\begin{Section}{Methods}
\describe{
\item[writeVolume] \code{signature(x="BrainVolume",fileName="character")} : writes volume to disk using default NIFTI format 
}
\end{Section}


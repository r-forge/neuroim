\HeaderA{IndexLookupVolume-class}{A class used to map from 1d indices to lookup table indices}{IndexLookupVolume.Rdash.class}
\aliasA{initialize,IndexLookupVolume-method}{IndexLookupVolume-class}{initialize,IndexLookupVolume.Rdash.method}
\aliasA{space,IndexLookupVolume-method}{IndexLookupVolume-class}{space,IndexLookupVolume.Rdash.method}
\keyword{classes}{IndexLookupVolume-class}
\begin{Description}\relax
An IndexLookupVolume maps from 1-d volume indices to an ordered set of indices that map to indices ina lookup table.  
This class is currently used in the implementation of \code{SparseBrainVector}
\end{Description}
\begin{Section}{Objects from the Class}
Objects can be created by calls of the form \code{new("IndexLookupVolume", space, indices)}.
\end{Section}
\begin{Section}{Slots}
\describe{
\item[\code{space}:] Object of class \code{"BrainSpace"} 
\item[\code{indices}:] Object of class \code{"integer"} 
\item[\code{map}:] Object of class \code{"integer"} 
}
\end{Section}
\begin{Section}{Methods}
\describe{
\item[coords] \code{signature(x = "IndexLookupVolume")}: ... 
\item[indices] \code{signature(x = "IndexLookupVolume")}: ... 
\item[initialize] \code{signature(.Object = "IndexLookupVolume")}: ... 
\item[lookup] \code{signature(x = "IndexLookupVolume", i = "numeric")}: ... 
\item[space] \code{signature(x = "IndexLookupVolume")}: ... 
}
\end{Section}
\begin{Examples}
\begin{ExampleCode}
showClass("IndexLookupVolume")
\end{ExampleCode}
\end{Examples}

